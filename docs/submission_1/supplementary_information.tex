% Options for packages loaded elsewhere
\PassOptionsToPackage{unicode}{hyperref}
\PassOptionsToPackage{hyphens}{url}
\documentclass[
  english,
  man]{apa6}
\usepackage{xcolor}
\usepackage{amsmath,amssymb}
\setcounter{secnumdepth}{-\maxdimen} % remove section numbering
\usepackage{iftex}
\ifPDFTeX
  \usepackage[T1]{fontenc}
  \usepackage[utf8]{inputenc}
  \usepackage{textcomp} % provide euro and other symbols
\else % if luatex or xetex
  \usepackage{unicode-math} % this also loads fontspec
  \defaultfontfeatures{Scale=MatchLowercase}
  \defaultfontfeatures[\rmfamily]{Ligatures=TeX,Scale=1}
\fi
\usepackage{lmodern}
\ifPDFTeX\else
  % xetex/luatex font selection
\fi
% Use upquote if available, for straight quotes in verbatim environments
\IfFileExists{upquote.sty}{\usepackage{upquote}}{}
\IfFileExists{microtype.sty}{% use microtype if available
  \usepackage[]{microtype}
  \UseMicrotypeSet[protrusion]{basicmath} % disable protrusion for tt fonts
}{}
\makeatletter
\@ifundefined{KOMAClassName}{% if non-KOMA class
  \IfFileExists{parskip.sty}{%
    \usepackage{parskip}
  }{% else
    \setlength{\parindent}{0pt}
    \setlength{\parskip}{6pt plus 2pt minus 1pt}}
}{% if KOMA class
  \KOMAoptions{parskip=half}}
\makeatother
% Make \paragraph and \subparagraph free-standing
\makeatletter
\ifx\paragraph\undefined\else
  \let\oldparagraph\paragraph
  \renewcommand{\paragraph}{
    \@ifstar
      \xxxParagraphStar
      \xxxParagraphNoStar
  }
  \newcommand{\xxxParagraphStar}[1]{\oldparagraph*{#1}\mbox{}}
  \newcommand{\xxxParagraphNoStar}[1]{\oldparagraph{#1}\mbox{}}
\fi
\ifx\subparagraph\undefined\else
  \let\oldsubparagraph\subparagraph
  \renewcommand{\subparagraph}{
    \@ifstar
      \xxxSubParagraphStar
      \xxxSubParagraphNoStar
  }
  \newcommand{\xxxSubParagraphStar}[1]{\oldsubparagraph*{#1}\mbox{}}
  \newcommand{\xxxSubParagraphNoStar}[1]{\oldsubparagraph{#1}\mbox{}}
\fi
\makeatother
\usepackage{longtable,booktabs,array}
\usepackage{calc} % for calculating minipage widths
% Correct order of tables after \paragraph or \subparagraph
\usepackage{etoolbox}
\makeatletter
\patchcmd\longtable{\par}{\if@noskipsec\mbox{}\fi\par}{}{}
\makeatother
% Allow footnotes in longtable head/foot
\IfFileExists{footnotehyper.sty}{\usepackage{footnotehyper}}{\usepackage{footnote}}
\makesavenoteenv{longtable}
\usepackage{graphicx}
\makeatletter
\newsavebox\pandoc@box
\newcommand*\pandocbounded[1]{% scales image to fit in text height/width
  \sbox\pandoc@box{#1}%
  \Gscale@div\@tempa{\textheight}{\dimexpr\ht\pandoc@box+\dp\pandoc@box\relax}%
  \Gscale@div\@tempb{\linewidth}{\wd\pandoc@box}%
  \ifdim\@tempb\p@<\@tempa\p@\let\@tempa\@tempb\fi% select the smaller of both
  \ifdim\@tempa\p@<\p@\scalebox{\@tempa}{\usebox\pandoc@box}%
  \else\usebox{\pandoc@box}%
  \fi%
}
% Set default figure placement to htbp
\def\fps@figure{htbp}
\makeatother
\ifLuaTeX
\usepackage[bidi=basic]{babel}
\else
\usepackage[bidi=default]{babel}
\fi
% get rid of language-specific shorthands (see #6817):
\let\LanguageShortHands\languageshorthands
\def\languageshorthands#1{}
\ifLuaTeX
  \usepackage[english]{selnolig} % disable illegal ligatures
\fi
\setlength{\emergencystretch}{3em} % prevent overfull lines
\providecommand{\tightlist}{%
  \setlength{\itemsep}{0pt}\setlength{\parskip}{0pt}}
% Manuscript styling
\usepackage{upgreek}
\captionsetup{font=singlespacing,justification=justified}

% Table formatting
\usepackage{longtable}
\usepackage{lscape}
% \usepackage[counterclockwise]{rotating}   % Landscape page setup for large tables
\usepackage{multirow}		% Table styling
\usepackage{tabularx}		% Control Column width
\usepackage[flushleft]{threeparttable}	% Allows for three part tables with a specified notes section
\usepackage{threeparttablex}            % Lets threeparttable work with longtable

% Create new environments so endfloat can handle them
% \newenvironment{ltable}
%   {\begin{landscape}\centering\begin{threeparttable}}
%   {\end{threeparttable}\end{landscape}}
\newenvironment{lltable}{\begin{landscape}\centering\begin{ThreePartTable}}{\end{ThreePartTable}\end{landscape}}

% Enables adjusting longtable caption width to table width
% Solution found at http://golatex.de/longtable-mit-caption-so-breit-wie-die-tabelle-t15767.html
\makeatletter
\newcommand\LastLTentrywidth{1em}
\newlength\longtablewidth
\setlength{\longtablewidth}{1in}
\newcommand{\getlongtablewidth}{\begingroup \ifcsname LT@\roman{LT@tables}\endcsname \global\longtablewidth=0pt \renewcommand{\LT@entry}[2]{\global\advance\longtablewidth by ##2\relax\gdef\LastLTentrywidth{##2}}\@nameuse{LT@\roman{LT@tables}} \fi \endgroup}

% \setlength{\parindent}{0.5in}
% \setlength{\parskip}{0pt plus 0pt minus 0pt}

% Overwrite redefinition of paragraph and subparagraph by the default LaTeX template
% See https://github.com/crsh/papaja/issues/292
\makeatletter
\renewcommand{\paragraph}{\@startsection{paragraph}{4}{\parindent}%
  {0\baselineskip \@plus 0.2ex \@minus 0.2ex}%
  {-1em}%
  {\normalfont\normalsize\bfseries\itshape\typesectitle}}

\renewcommand{\subparagraph}[1]{\@startsection{subparagraph}{5}{1em}%
  {0\baselineskip \@plus 0.2ex \@minus 0.2ex}%
  {-\z@\relax}%
  {\normalfont\normalsize\itshape\hspace{\parindent}{#1}\textit{\addperi}}{\relax}}
\makeatother

\makeatletter
\usepackage{etoolbox}
\patchcmd{\maketitle}
  {\section{\normalfont\normalsize\abstractname}}
  {\section*{\normalfont\normalsize\abstractname}}
  {}{\typeout{Failed to patch abstract.}}
\patchcmd{\maketitle}
  {\section{\protect\normalfont{\@title}}}
  {\section*{\protect\normalfont{\@title}}}
  {}{\typeout{Failed to patch title.}}
\makeatother

\usepackage{xpatch}
\makeatletter
\xapptocmd\appendix
  {\xapptocmd\section
    {\addcontentsline{toc}{section}{\appendixname\ifoneappendix\else~\theappendix\fi: #1}}
    {}{\InnerPatchFailed}%
  }
{}{\PatchFailed}
\makeatother
\keywords{keywords\newline\indent Word count: X}
\DeclareDelayedFloatFlavor{ThreePartTable}{table}
\DeclareDelayedFloatFlavor{lltable}{table}
\DeclareDelayedFloatFlavor*{longtable}{table}
\makeatletter
\renewcommand{\efloat@iwrite}[1]{\immediate\expandafter\protected@write\csname efloat@post#1\endcsname{}}
\makeatother
\usepackage{lineno}

\linenumbers
\usepackage{csquotes}
\usepackage{bookmark}
\IfFileExists{xurl.sty}{\usepackage{xurl}}{} % add URL line breaks if available
\urlstyle{same}
\hypersetup{
  pdftitle={The title},
  pdfauthor={First Author1 \& Ernst-August Doelle1,2},
  pdflang={en-EN},
  pdfkeywords={keywords},
  hidelinks,
  pdfcreator={LaTeX via pandoc}}

\title{The title}
\author{First Author\textsuperscript{1} \& Ernst-August Doelle\textsuperscript{1,2}}
\date{}


\shorttitle{Title}

\authornote{

Add complete departmental affiliations for each author here. Each new line herein must be indented, like this line.

Enter author note here.

The authors made the following contributions. First Author: Conceptualization, Writing - Original Draft Preparation, Writing - Review \& Editing; Ernst-August Doelle: Writing - Review \& Editing, Supervision.

Correspondence concerning this article should be addressed to First Author, Postal address. E-mail: \href{mailto:my@email.com}{\nolinkurl{my@email.com}}

}

\affiliation{\vspace{0.5cm}\textsuperscript{1} Wilhelm-Wundt-University\\\textsuperscript{2} Konstanz Business School}

\abstract{%
One or two sentences providing a \textbf{basic introduction} to the field, comprehensible to a scientist in any discipline.
Two to three sentences of \textbf{more detailed background}, comprehensible to scientists in related disciplines.
One sentence clearly stating the \textbf{general problem} being addressed by this particular study.
One sentence summarizing the main result (with the words ``\textbf{here we show}'' or their equivalent).
Two or three sentences explaining what the \textbf{main result} reveals in direct comparison to what was thought to be the case previously, or how the main result adds to previous knowledge.
One or two sentences to put the results into a more \textbf{general context}.
Two or three sentences to provide a \textbf{broader perspective}, readily comprehensible to a scientist in any discipline.
}



\begin{document}
\maketitle

\section{Main Effects of Exam Stress on Vocal Production}\label{main-effects-of-exam-stress-on-vocal-production}

\subsection{F0}\label{f0}

Between-person variability in baseline F0 was substantial (τ₁ = {[}VALUE{]} Hz, 95\% CI
{[}LOWER, UPPER{]}), accounting for 83\% of total variance (ICC = 0.83). Individual
differences in stress reactivity were moderate (τ₂ = {[}VALUE{]} Hz, 95\% CI {[}LOWER, UPPER{]}),
with a τ₂/β₁ ratio of 0.33, indicating that individual variability was approximately
one-third the magnitude of the average stress effect. Importantly, using the posterior
distribution of random slopes, we estimated that 95\% of individuals exhibited
stress-related F0 increases ranging from 1.2 to 5.4 Hz, indicating that while the
magnitude varied across individuals, the direction of the effect was consistent:
all participants showed pitch elevation under stress.

\subsection{NNE}\label{nne}

Between-person variability in baseline NNE was substantial (τ₁ = {[}VALUE{]} dB, 95\% CI
{[}LOWER, UPPER{]}), accounting for 55\% of total variance (ICC = 0.55). Critically,
individual differences in stress reactivity were very large (τ₂ = {[}VALUE{]} dB, 95\% CI
{[}LOWER, UPPER{]}), with a τ₂/β₁ ratio of 1.27, indicating that individual heterogeneity
exceeded the magnitude of the average population effect.

Using the posterior distribution of random slopes, we estimated that 95\% of individuals
exhibited stress-related NNE changes ranging from -2.3 to +1.0 dB. This wide prediction
interval crosses zero, revealing substantial heterogeneity in individual responses.
Specifically, we estimate that approximately 22\% of individuals showed stress-related
increases in glottal noise (positive NNE changes), contrary to the population-level
trend toward noise reduction. This pronounced between-person heterogeneity contrasts
sharply with the consistency of F0 responses, where all individuals showed pitch
elevation under stress, and suggests that phonatory control adjustments are more
strongly influenced by individual differences than arousal-driven pitch changes.

\begin{longtable}[]{@{}
  >{\raggedright\arraybackslash}p{(\linewidth - 4\tabcolsep) * \real{0.3636}}
  >{\raggedright\arraybackslash}p{(\linewidth - 4\tabcolsep) * \real{0.4091}}
  >{\raggedright\arraybackslash}p{(\linewidth - 4\tabcolsep) * \real{0.2273}}@{}}
\toprule\noalign{}
\begin{minipage}[b]{\linewidth}\raggedright
Metric
\end{minipage} & \begin{minipage}[b]{\linewidth}\raggedright
F0 Mean
\end{minipage} & \begin{minipage}[b]{\linewidth}\raggedright
NNE
\end{minipage} \\
\midrule\noalign{}
\endhead
\bottomrule\noalign{}
\endlastfoot
Population effect (β₁) & +3.3 Hz*** & -0.65 dB*** \\
Between-person SD (τ₂) & 1.1 Hz & 0.83 dB \\
Ratio τ₂/β₁ & 0.33 & 1.27 \\
ICC (baseline) & 82.6\% & 54.5\% \\
95\% PI for individual effect & {[}1.2, 5.4{]} Hz & {[}-2.3, +1.0{]} dB \\
\% with opposite effect & 0\% & 21.6\% \\
\textbf{Interpretation} & \textbf{Universal response} & \textbf{Heterogeneous response} \\
\end{longtable}

\subsection{Individual Variability: Universal F0 Elevation vs Heterogeneous NNE Responses}\label{individual-variability-universal-f0-elevation-vs-heterogeneous-nne-responses}

\subsubsection{Contrasting Patterns of Individual Differences}\label{contrasting-patterns-of-individual-differences}

A critical finding emerged from variance partitioning: stress-induced F0 elevation was
remarkably consistent across individuals (τ₂/β₁ = 0.33; 95\% PI: 1.2--5.4 Hz, all positive),
whereas NNE responses showed pronounced heterogeneity (τ₂/β₁ = 1.27; 95\% PI: -2.3 to +1.0 dB,
crossing zero). This fourfold difference in relative variability indicates fundamentally
different processes underlying these two dimensions of vocal stress response.

Most strikingly, while 100\% of individuals exhibited stress-related pitch elevation, our
model estimates that approximately 22\% showed stress-related increases in glottal noise---
directly opposite to the population-level trend toward reduced noise. This is not merely
statistical noise: individuals showing ``paradoxical'' NNE increases may represent a distinct
subgroup employing qualitatively different phonatory strategies under stress.

\subsubsection{Theoretical Implications}\label{theoretical-implications}

\textbf{Why is F0 universal but NNE heterogeneous?}

The universality of F0 elevation likely reflects the obligatory nature of autonomic stress
responses. Sympathetic activation increases laryngeal muscle tension and subglottal pressure
via relatively automatic, phylogenetically conserved pathways that operate similarly across
individuals. While the magnitude varies (presumably reflecting trait differences in
autonomic reactivity), the direction is invariant.

In contrast, NNE---indexing the balance between harmonic and noise energy in the glottal
signal---depends on fine-grained adjustments in vocal fold contact patterns, adduction force,
and phonatory mode. These adjustments likely reflect a combination of:

\begin{enumerate}
\def\labelenumi{\arabic{enumi}.}
\item
  \textbf{Compensatory vocal strategies}: Some speakers may habitually adopt pressed phonation
  under stress (reducing noise), while others maintain modal phonation or even shift toward
  breathy voice (increasing noise)
\item
  \textbf{Baseline vocal habits}: Individuals with habitually pressed or tense voice quality may
  show ceiling effects, limiting further noise reduction
\item
  \textbf{Attentional modulation}: Stress-induced hypervigilance might promote conscious vocal
  monitoring in some individuals (enhancing control, reducing noise) but distraction in
  others (degrading control, increasing noise)
\item
  \textbf{Physiological constraints}: Anatomical differences in laryngeal structure or chronic
  vocal tension may constrain the range of phonatory adjustments available under stress
\end{enumerate}

\subsubsection{Practical and Clinical Implications}\label{practical-and-clinical-implications}

These findings have important implications for applied stress detection:

\textbf{F0 as a population-level biomarker}: The consistency of stress-related F0 elevation across
all individuals---despite variation in magnitude---suggests F0 could serve as a relatively
robust, calibration-free indicator of acute stress in real-world applications.

\textbf{NNE requires individual calibration}: The 22\% of individuals showing opposite effects means
NNE cannot be interpreted without individual baseline data or additional contextual information.
This may explain why personality trait moderators (examined separately) showed stronger effects
for F0 than NNE: traits may predict magnitude of universal responses more reliably than
direction of heterogeneous responses.

\textbf{Multi-parameter approaches essential}: The dissociation underscores that vocal stress
responses are multidimensional. Relying solely on NNE (or any single parameter) risks
misclassifying the \textasciitilde22\% of individuals whose responses deviate from population trends.
Combining F0 (universal) with NNE (heterogeneous but informative about individual strategies)
provides complementary information about both general arousal state and individual-specific
regulatory dynamics.

\subsubsection{Unanswered Questions}\label{unanswered-questions}

Why do \textasciitilde22\% of individuals show increased glottal noise under stress? Potential explanations
include:

\begin{itemize}
\tightlist
\item
  \textbf{Vocal fatigue}: If stress-induced tension leads to incomplete glottal closure in some
  individuals
\item
  \textbf{Breathy phonation as anxiety marker}: Some individuals may adopt breathy voice as a
  submissive or affiliative signal under evaluative stress
\item
  \textbf{Measurement confounds}: Increased respiratory rate under stress might introduce
  turbulent noise that NNE captures but that reflects respiration rather than phonation
\end{itemize}

Future work should:
1. Examine whether the 22\% showing paradoxical NNE increases differ systematically in
personality, vocal training, or baseline voice quality
2. Use laryngeal imaging or EGG to directly observe glottal dynamics in high vs low NNE
responders
3. Test whether individual NNE response patterns are stable across stressors or represent
state-dependent variation

\newpage

\section{References}\label{references}

\phantomsection\label{refs}


\end{document}
